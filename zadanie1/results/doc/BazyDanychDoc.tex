\documentclass[polish, 11pt]{article}
    
\usepackage[a4paper, margin=25mm]{geometry}
\usepackage{babel,polski}
\usepackage[utf8]{inputenc}
\usepackage[T1]{fontenc}
\usepackage{booktabs,multirow,multicol}

\usepackage{graphicx}
\graphicspath{{img/}}

\usepackage{xcolor}
\usepackage[font=small,labelfont=bf]{caption}

%\newcommand{\results}[3][1.0]{
%    \includegraphics[width=#1\textwidth]{#2}
%    \captionof{figure}{#3\label{fig:#2}}
%}

\captionsetup[figure]{name=Rys.}

\newcommand{\obrazek}[3][1.0]{
    \includegraphics[width=#1\textwidth]{#2}
    \captionof{figure}{#3\label{fig:#2}}
}

\begin{document}

\begin{titlepage}
    \centering
    {\scshape\LARGE  Bazy Danych \\ projekt \par}
    \vspace{1cm}
    {\scshape\Large ,,System rankingowy dla sportu drużynowego''\par}
    \vspace{2cm}
 
    Grupa 08\par
    {\itshape\Large Jacek Czewonka (lider) --- 241373\/\par}
    {\itshape\Large Konrad Lewandowski --- 241326\/\par}
    \vfill
    Prowadzący:\par
    mgr inż. Karol Puchała

    \vfill

    {\large Wrocław, \today\par}

\end{titlepage}

\tableofcontents
\newpage

\section{Wstęp}
	Celem projektu jest zaprojektowanie bazy danych dla systemu rankingowego. Obiektem modelowanym będą rozgrywki sportu zespołowego Ultimate Frisbee.

\section{Model}
    \subsection{Opis świata rzeczywistego }
	    Modelowanym  obiektem są wyniki rozgrywek Ultimate Frisbee. W przeciągu lat wykształciło się specyficzne środowisko graczy. Mecze i turnieje rozgrywane są w trzech kategoriach: męskiej, kobiecej i mieszanej. Kategoria mieszana jest najpopularniejszą, lecz pozostałe kategorie silnie się rozwijają. W sezonie letnim gra się na trawie 7 na 7, w sezonie zimowym na hali 5 na 5.
		 \subsection{Model bazy danych }
		 	Relacyjna baza danych przedstawia zależności miedzy: zawodnikami,  drużynami, meczami oraz turniejami.

	\subsection{Tabele}
	    
	    \begin{itemize}
	    
	    	\item 	 Zawodnicy:
	    		\begin{itemize}
	    			\item ID\_zawodnika (Primary Key)
			    	\item Imie
			    	\item Nazwisko
			    	\item Wiek
			    	\item Płeć
			    	\item ID\_drużyny (Forein Key)
	    		\end{itemize}
	    		
	    	\item 	   Drużyny:
	    		\begin{itemize}
			    	\item ID\_drużyny (Primary Key)
					\item nazwa\_drużyny
					\item skrót
					\item kategoria
					\item miasto\_pochodzenia
				 \end{itemize}
				 
	    	\item	Turnieje:
	    		\begin{itemize}
			    	\item ID\_turnieju (Primary Key)
			    	\item nazwa\_turnirju
			    	\item kategoria
			    	\item data
			    	\item uczestnicy(teams) (Forein Key)
			    	\item ranga
			    \end{itemize}

	    	\item	Mecze:
	    		\begin{itemize}
			    	\item ID\_meczu (Primary Key)
			    	\item ID\_turnieju (Forein Key)
			    	\item team\_A (Forein Key)
			    	\item team\_B (Forein Key)
			    	\item data
			    	\item wynik
			    \end{itemize}
			    
	    	\item	Punkty:
	    		\begin{itemize}
			    	\item ID\_punktu (Primary Key)
			    	\item czas
			    	\item wynik (Forein Key)
			    	\item asysta (Forein Key)
			    	\item punktujący
			    	\item ID\_meczu (Forein Key)
			    \end{itemize}
			    	  
	    \end{itemize}
	    
	\section{Opis funkcjonalny}
	
		\subsection{Opis słowny}
			
				\subparagraph{ Przeglądanie zawartości bazy.\\} 
					System będzie pozwalał każdej zainteresowanej osobie na wgląd do zawartości bazy. Kluczowym elementem jest transparentność systemu. Jedynie dane poufne dotyczące zawodników będą ukryte.
				\subparagraph{Edytowanie drużyn. \\}
					Managerowie drużyn dostaną możliwość wprowadzania zmian w składach oraz reszcie danych swoich zespołów.
				\subparagraph{Organizacja turniejów \\}
					Użytkownicy ze statusem Manager Turnieju będą mogli tworzyć turnieje, modyfikować wszystkie dane w zakresie zgłoszonego turnieju, a następnie je udostępniać.
				\subparagraph{Administrowanie bazą \\}
					Administrator ma nieograniczony dostęp do bazy oraz możliwość edytowanie jej. Po każdym sezonie letnim i zimowym baza powinna być archiwizowana. W przypadku konfliktów (np. dwie osoby podające się za managerów tej samej drużyny wprowadzą sprzeczne dane) administrator ingeruje w celu zachowania spójności danych.
			
		
		
		\subsection{Use case diagram}
			\begin{center}
				\obrazek[0.9]{BD-UseCase2popr.png}{use case diagram}
			\end{center}
		



		    
		    \vspace{0.5cm}



\end{document}
